\documentclass[a4paper, 10pt]{article}

\usepackage[utf8]{inputenc}
\usepackage[italian]{babel}
\usepackage[T1]{fontenc}
\usepackage{csquotes}

\usepackage[colorlinks=true]{hyperref}

\usepackage{datetime2}
\usepackage{enumitem}

% change datetime2 settings
\DTMsetdatestyle{ddmmyyyy}
\DTMsetup{datesep=/}

\usepackage{biblatex}
\addbibresource{citations.bib}


\begin{document}

\title{Risolutore di sudoku in Python con OpenCV e Keras}
\author{Simone Fidanza}
\date{\today}


\maketitle

\begin{abstract}
    Ancora vuoto...
\end{abstract}


\section{Introduzione}\label{sec:introduzione}

L'obiettivo ultimo dell'applicazione è quello di risolvere un sudoku da
un'immagine. Per poter fare ciò sono necessari diversi passaggi:

\begin{enumerate}
    \item Fornire un'immagine in input contenente un sudoku;
    \item Localizzare la griglia del sudoku e estrarla;
    \item Data la griglia, individuare ogni singola cella della stessa;
    \item Determinare se la cella contiene una cifra e se sì effettuare il
        riconoscimento ottico dei caratteri (per brevità OCR);
    \item Risolvere il sudoku mediante un algoritmo;
    \item Mostrare il sudoku risolto all'utente.
\end{enumerate}

L'applicazione fa ampio utilizzo delle librerie OpenCV\footnote{acronimo in
lingua inglese di \emph{Open Source Computer Vision Library}
}~\cite{opencv_library}, TensorFlow~\cite{tensorflow2015-whitepaper},
Keras~\cite{chollet2015keras} e NumPy~\cite{harris2020array}.

Per poter effettuare l'OCR è necessario un modello di Machine Learning.


\section{Modello}

Il modello di Machine Learning proposto è un miglioramento della rete neurale
LeNet5~\cite{lecun1998gradient}



\bigskip
\printbibliography

\end{document}
