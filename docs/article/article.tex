\documentclass[a4paper, 10pt]{article}

\usepackage[utf8]{inputenc}
\usepackage[italian]{babel}
\usepackage[T1]{fontenc}
\usepackage{csquotes}

\usepackage[colorlinks=true]{hyperref}

\usepackage{datetime2}
\usepackage{enumitem}

\usepackage{biblatex}
\addbibresource{citations.bib}


\begin{document}

\title{Risolutore di sudoku in Python con OpenCV e Keras}
\author{Simone Fidanza}
\date{\today}


\maketitle

\begin{abstract}
    Ancora vuoto...
\end{abstract}


\section{Introduzione}\label{sec:introduzione}

Il risolutore di sudoku (pydoku) è scritto in Python e fa largo utilizzo di
librerie come OpenCV~\cite{opencv_library} (\emph{Open source Computer
Vision}), TensorFlow~\cite{tensorflow2015-whitepaper},
Keras~\cite{chollet2015keras} e NumPy~\cite{harris2020array}.





\section{Modello}

\medskip

\printbibliography

\end{document}
